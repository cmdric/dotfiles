\documentclass[compress,10pt]{beamer} 
\usepackage{mathptmx}
\usepackage{helvet}
\usepackage{/home/potterat/Documents/latex_bib_sty/include_pres}
\usepackage{lmodern}
\usepackage[T1]{fontenc}
\newcommand{\myauth}{{}}
\newcommand{\mytitle}{\tiny{s21.jets}}
\setbeamertemplate{navigation symbols}{}
\usetheme[dark,framenumber]{CedricUB}
\usepackage{keyval}


\usepackage{tikz}
\usetikzlibrary{shapes,snakes}

\usepackage{verbatim}
\newcommand\putpic[5]{% 
    \begin{textblock}{#3}(#1,#2) 
        \includegraphics[width=#3, 
        height=#4]{#5} 
    \end{textblock} 
} 

\newcounter{backupcounter}

\title{\vspace{-1.0cm}Jets in Stripping\\
    \vspace{2.7cm}
    \small{C\'edric Potterat}
}
%\subtitle{\vspace{1.45cm}\small{Reun\~{a}o LAPE\vspace{0.7cm}}}
\author{\vspace{4.3cm}\scriptsize{17/06/2014}\hfill\small{{Jets Reco.}}}





\begin{document}

% ---------------------------------------------------------------------------
% *** Titlepage <<<

% ----------------------------------------------------------------------------
\maketitle
% ----------------------------------------------------------------------------
% *** END of Titlepage >>>
% ----------------------------------------------------------------------------




\begin{frame}
    \frametitle{reminder - status} 
    \begin{block}{}
        \begin{itemize}
            \item StdParticleFlow and StdJets  (R=0.5) are in CommonParticle 
            \item usage via normal DoD.
            \item it takes time to run both ! cutting on other variables before may be needed to be mandatory. 
        \end{itemize}
    \end{block}

     \begin{block}{StdJets}
        \begin{itemize}
            \item call by default StdParticleFlow (default config)
            \item anti-$k_T$, E scheme and R=0.5, 5GeV cut on $p_T$
            \item all its constituants are saved as daugthers
            \item call the new JEC - Reco14
        \end{itemize}
    \end{block}
    \begin{block}{need to be redone}
        \begin{itemize}
            \item (not improtant for the stripping, but the output location of StdParticleFlow)
            \item if in an extrem case you want to use custom jets\ldots do not use names like: "StdJets","PF", "ParticleFlow", "StdParticleFlow" nor "PFParticles". it could create conflics.
        \end{itemize}
    \end{block}
\end{frame}

\begin{frame}
    \frametitle{S21}  
    \begin{block}{}
        \begin{itemize}
            \item R=0.7 in CommonParticle, open to discussion ?
        \end{itemize}
    \end{block}

    \begin{block}{RAW, mdst}
        \begin{itemize}
            \item If you do not need to do fancy stuff with RAW, it is not needed after the ParticleFlow.
            \item If you want to run btagging after the stripping (since it won't be in) it is safer to ask for full DST.
            \item but, all remeber that anyway all particle that are constituante of a jet, are saved as daugthers. so if you intend to btag only no jet constituant you could go for mdst.
            \item after removing the RAW info I do not explect to have a large difference between mdst and dst in the jet cases, we do save a lot of particles.
        \end{itemize}
    \end{block}
    \begin{block}{}
I recommand to ask for dst without raw information.
\end{block}
\end{frame}
\end{document}
